% main.tex
\documentclass[10pt,twocolumn]{article}
% Se preferisci stile IEEE/commenti per conferenza, sostituisci la classe con:
% \documentclass[conference]{IEEEtran}

% -------- PACKAGES --------
\usepackage[utf8]{inputenc}
\usepackage[T1]{fontenc}
\usepackage[italian, english]{babel} % usa 'italian' come lingua principale se vuoi
\usepackage{microtype}
\usepackage{lmodern}

\usepackage{amsmath,amssymb,amsthm}
\usepackage{graphicx}
\usepackage{caption,subcaption}
\usepackage{booktabs}       % tabelle belle
\usepackage{siunitx}        % unità e numeri
\usepackage{enumitem}
\usepackage{url}
\usepackage{hyperref}       % link cliccabili
\usepackage{cleveref}       % riferimenti intelligenti (\cref)
\usepackage{float}          % [H] per figure
\usepackage{algorithm}
\usepackage{algorithmic}    % o algorithmicx se preferisci
\usepackage{listings}       % per codice
\usepackage{xcolor}

% Margini (modifica se la conferenza richiede altro)
\usepackage[margin=1.8cm]{geometry}

% ---------- MACROS ----------
\newcommand{\ie}{i.e.\ }
\newcommand{\eg}{e.g.\ }
\newcommand{\todo}[1]{\textbf{\textcolor{red}{TODO: #1}}}

% Theorem environment (esempi)
\theoremstyle{definition}
\newtheorem{definition}{Definition}
\theoremstyle{plain}
\newtheorem{theorem}{Theorem}

% ---------- METADATA ----------
\title{Practical Implementation of Deep Differentiable Logic Gate Networks:  \\
    Design and Benchmarks}
\author{
  \begin{tabular}{c}
    Author 1 \\
    Affiliation
  \end{tabular}
  \and
  \begin{tabular}{c}
    Author 2 \\
    Affiliation
  \end{tabular}
  \and
  \begin{tabular}{c}
    Author 3 \\
    Affiliation
  \end{tabular}\and
  \begin{tabular}{c}
    Author 4 \\
    Affiliation
  \end{tabular}
}
\date{\today}

% ---------- DOCUMENT ----------
\begin{document}
\twocolumn[
  \maketitle
  \begin{center}
    \begin{minipage}{0.92\textwidth}
      \begin{abstract}
        Write abstract here
        \cite{petersen2022deepdifferentiablelogicgate}
      \end{abstract}
      \vspace{0.2cm}
    \end{minipage}
  \end{center}
  \medskip
]

\section{Introduction}
The introduction section should:
\begin{itemize}[noitemsep]
  \item contextualize the problem and explain why it is important;
  \item clarify the limitations of existing solutions;
  \item state the main contributions (with clear bullet points);
  \item provide a brief overview of the paper structure.
\end{itemize}

\paragraph{Contributions} List 2--4 main contributions clearly.

\section{Related Work}
Critical review of related work; organize by themes (e.g.: models, architectures, metrics, applications).
Cite works using \verb|\citep| or \verb|\citet| (with \texttt{natbib}; see the attached .bib file).

\section{Method}
Describe the proposed method in detail.
Use subsections for clarity.

\subsection{Problem formulation}
Mathematical statement of the problem, notation and assumptions.

\subsection{Architecture / Algorithm}
Present the architecture (diagrams) or the algorithm in pseudo-code.

\begin{algorithm}[H]
  \caption{Example algorithm}
  \begin{algorithmic}[1]
    \REQUIRE dataset $\mathcal{D}$, hyperparameters $\theta$
    \ENSURE trained model $M$
    \FOR{each epoch $e=1,\dots,E$}
      \FOR{batch $b$ in $\mathcal{D}$}
        \STATE compute loss $\mathcal{L}$
        \STATE update parameters via optimizer
      \ENDFOR
    \ENDFOR
  \end{algorithmic}
\end{algorithm}

\section{Experimental Setup}
Provide details on:
\begin{itemize}[noitemsep]
  \item dataset (source, splits, preprocessing);
  \item evaluation metrics;
  \item architectures and hyperparameters;
  \item experimental environment (GPU, seed, library versions).
\end{itemize}
Reproducibility: indicate whether the code/data are public (link to repo).

\section{Results}
Present quantitative and qualitative results.

\subsection{Quantitative results}
\begin{table}[H]
  \centering
  \caption{Main results on dataset X.}
  \begin{tabular}{lccc}
    \toprule
    Method & Accuracy (\%) & F1 & Time (s) \\
    \midrule
    Baseline A & 85.2 & 0.84 & 120 \\
    Proposed Method & \textbf{89.7} & \textbf{0.88} & 150 \\
    \bottomrule
  \end{tabular}
\end{table}

\subsection{Qualitative results}
Include visual examples, heatmaps or use cases.

\begin{figure}[H]
  \centering
  \begin{subfigure}[b]{0.45\linewidth}
    \includegraphics[width=\linewidth]{example-image-a}
    \caption{Example A}
  \end{subfigure}
  \hfill
  \begin{subfigure}[b]{0.45\linewidth}
    \includegraphics[width=\linewidth]{example-image-b}
    \caption{Example B}
  \end{subfigure}
  \caption{Qualitative examples of the proposed model.}
  \label{fig:examples}
\end{figure}

\section{Ablation Study}
Analyze key model components to show what contributes to performance.

\section{Discussion}
Interpretation of results, limitations, ethical considerations if relevant, possible applications and impact.

\section{Conclusion and Future Work}
Summary of contributions and directions for future work.

\section*{Acknowledgments}
Acknowledgments to organizations, funding and collaborators.

% -------- BIBLIOGRAPHY ----------
% Use natbib for author-year or numeric citations depending on the style.
% Compilation: pdflatex main.tex; bibtex main; pdflatex main.tex; pdflatex main.tex

\bibliographystyle{plain} % We choose the "plain" reference style
\bibliography{main} % Entries are in the refs.bib file

% ---------- APPENDIX ----------
\appendix
\section{Implementation details}
Parameters, additional pseudocode, mathematical proofs.

\section{Supplementary results}
Additional plots and tables.

\end{document}
